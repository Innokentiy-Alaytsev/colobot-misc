\section{Game modes}

\subsection{Missions}

The missions are the core aspect of the whole program. In order to get access to the next mission, you have got to succeed in the former mission. However, in order to avoid players getting stuck in one mission and getting angry at the game, a ''solution'' text explains the necessary steps to succeed; this text is available after you died three times.

The story developed in the whole cycle of missions can be summarized as follows: A first expedition looking for new planets to colonize left earth eight month ago. This mission was composed only of robot, and met various problems, including transmission problems, until Houston completely lost contact with it.

We know little about what this first mission managed to achieve, whether they found an adequate planet or not. It is time for a second mission to take off, with a man on board: You were chosen to be that man.

You fly from planet to planet, facing problems, at the same time developing new technologies thanks to own research and to the research done by the first expedition. Fighting against alien hostile insects is one important task.

However, the initial ''shoot on everything that moves'' attitude faces more and more ecological and political pressure, that results in strategy changes: Offensive weapons are more and more replaced by defensive systems. In parallel to this evolution, more and more of the ''terrestrian'' technology is replaced by a new technology integrating some ''green'' organic elements issued from aliens.


\subsection{Free game}

% no description in original Colobot GDD
Free game is a type of sandbox gameplay. You are placed on a planet of your choice with some limited number of titanium cubes and power cells. There are no mission objectives. You can do whatever you want, build a base, test your programs and have fun in general.

\todo[inline,color=green]{Future plans: Randomly generate map and ores on the ground, according to user's choice}


\subsection{Exercises}

In the exercises, the player can learn the fundamentals of programming. However, the pedagogical concept that underlies the exercises as they exist now has evolved to adapt to the newest results in pedagogical research, so many exercises will have to be made new. For this reason most of the exercises were not translated to English, the few that were may give you an idea, knowing that we intend to replace the exercises by new ones that take up all the tasks encountered during the missions, and take the programs transmitted by Houston as a basis.


\subsection{Challenges}

% no description in original Colobot GDD
\todo[inline]{no description in original Colobot GDD}

% The challenges have not been created yet. They require that the player did at least a part of the exercises, and will probably interest only a minority of the players. However, this part of COLOBOT can develop an interesting dynamic, if some players start to develop interesting programs, that could be published on the COLOBOT website.