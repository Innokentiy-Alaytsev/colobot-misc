\section{Houston Mission Control}

Earth control center for all your space missions. Though still named after the well-known mission control of the old days, ''Houston'' is actually located in the middle of the Nevada Desert, right where it's all starting again... with you on-board this time.

\begin{description}
    \item[Requirement for building:] Cannot be built
    \item[Object category:] \texttt{Houston}
\end{description}


\section{Space Ship}

Your means of transportation from one planet to the next, the only way to travel safely across the cosmos and bring success to your missions. In addition to yourself, the spaceship can carry bots and raw materials. Whenever a mission is completed, you must select the spaceship, then click on the <takeoff> command.

\begin{description}
    \item[Requirement for building:] Cannot be built
    \item[Object category:] \texttt{SpaceShip}
\end{description}


\section{Bot Factory}

This building is intended for the manufacturing of bots using titanium.

\begin{enumerate}
    \item Place the titanium inside the factory,
    \item Step back out,
    \item Select the factory,
    \item Click on the button showing the diagram of the bot you want the factory to assemble.
\end{enumerate}

The finished bot does not include an on-board power cell. You'll need to supply it with one for it to be able to leave the factory. The list of feasible bots will depend upon the research your research center has completed at this time.

\begin{description}
    \item[Requirement for building:] any flat surface at least 7.5-meter wide
    \item[Input:] Titanium
    \item[Output:] Completed bot
    \item[Power source:] None
    \item[Duration of operation:] 15 seconds
    \item[Object category:] \texttt{BotFactory}
\end{description}


\section{Research Center}

The research center is an enormous computer. Its most useful feature is its ability to come up with new technologies as well as improvements on existing ones regarding buildings and bots in particular. The 65,536 processors it contains use up a large amount of energy. Each research necessitates a brand new and fully charged power cell.

\begin{description}
    \item[Requirement for building:] Any flat surface at least 5-meter wide
    \item[Input:] Request for research
    \item[Output:] New and improved technologies
    \item[Power source:] Regular or nuclear power cell
    \item[Duration of operation:] Between 9 seconds and 1 minute depending on the research
    \item[Object category:] \texttt{ResearchCenter}
\end{description}


\section{Radar Station}

The radar designates, among other things, the position of the nearest enemies. If they are within the \todo{I don't think radars have range. Maybe they should?}radar's range, they'll appear on your mini-map at the bottom right corner of your screen. Radars are also vital relay stations for the transmission of information to and from the Earth: on every planet, a radar station must be left behind, making sure that it will not be damaged by aliens. 

\begin{description}
    \item[Requirement for building:] Any flat surface at least 2.5-meter wide
    \item[Input:] None
    \item[Output:] None
    \item[Power source:] None
    \item[Duration of operation:] Constant
    \item[Object category:] \texttt{RadarStation}
\end{description}


\section{Exchange Post}

This building delivers digital information; it is used only in some programming exercises. A post can contain up to 10 pieces of information, each one referenced by name. To obtain information, a bot must be close enough to the post (less than 12 meters) and execute the command receive.

\begin{description}
    \item[Requirement for building:] Any flat surface at least 2.5-meter wide
    \item[Input:] None
    \item[Output:] Digital information
    \item[Power source:] None
    \item[Duration of operation:] 1 second
    \item[Object category:] \texttt{ExchangePost}
\end{description}


\section{Repair Center}

The repair center regenerates the shielding on damaged bots. A bot's shielding will absorb a certain amount of enemy hits. Nothing left of the shielding means that the bot itself, when hit next, will be destroyed.

\begin{description}
    \item[Requirement for building:] Any flat surface at least 6-meter wide
    \item[Input:] Damaged bot
    \item[Output:] Repaired bot with regenerated shielding
    \item[Power source:] None
    \item[Duration of operation:] A few seconds depending on the repair job to be done
    \item[Object category:] \texttt{RepairCenter}
\end{description}


\section{Defense Tower}

The tower constitutes the best defense against enemy attacks, whether they originate on the ground or from the skies. The tower requires either a regular or a nuclear power cell. A regular power cell provides a capacity for 8 shots. A nuclear power cell is of course preferable. The tower starts blinking when it runs out of power. The range is 35 meters. To visualize it, select the tower then hit the <range> button. Red dots outline the circular zone for 20 seconds.

\begin{description}
    \item[Requirement for building:] any flat surface at least 3.5-meter wide
    \item[Power source:] regular or nuclear power cell
    \item[Duration of operation:] 1 second to aim the cannon. \todo{I don't remember that...}A preliminary research may offer an improvement in promptness, therefore more efficiency against flying targets
    \item[Object category:] \texttt{DefenseTower}
\end{description}


\section{Auto Lab}

The lab is intended for the analysis of organic matter. It will help you familiarize with the insects' very own technology and perhaps even use it to your advantage.

\begin{description}
    \item[Requirement for building:] Any flat surface at least 6-meter wide
    \item[Input:] Organic matter
    \item[Output:] New and improved technologies
    \item[Power source:] None
    \item[Duration of operation:] 20 seconds
    \item[Object category:] \texttt{AutoLab}
\end{description}


\section{Power Station}

The power station extracts energy from the underground and recharges regular power cells through induction. In order to determine the best site for a power station to be built on, a sniffer should be used to prospect the subsoil. The marks it lays down provide information about the specifics of its findings: a green cross means that there is an energy deposit at this location beneath the surface. The energy deposit is indispensable to the construction of a power station or a power plant. The large power cell at the top of the station remaining red (empty) can only mean that the site is not geologically adequate. To recharge a power cell loaded at the back of a bot, just move the bot to the center of the platform and wait a few seconds. A power cell carried at arm's length can also be recharged. Nuclear power cells are non-rechargeable.

\begin{description}
    \item[Requirement for building:] Any flat and geologically adequate surface at least 6-meter wide
    \item[Input:] Empty power cell
    \item[Output:] Recharged power cell
    \item[Power source:] Subsoil energy
    \item[Duration of operation:] A few seconds, depending on the available energy
    \item[Object category:] \texttt{PowerStation}
\end{description}


\section{Power Plant}

% no description in original Colobot GDD

The power plant is power cell factory. It transforms a chunk of titanium into a fully charged power cell. It requires an underground energy source in place of construction in order to function.

\begin{description}
    \item[Requirement for building:] Any flat surface at least \todo{Check}?-meter wide
    \item[Input:] Titanium
    \item[Output:] Full power cell
    \item[Power source:] None
    \item[Duration of operation:] \todo{Check}? seconds
    \item[Object category:] \texttt{PowerPlant}
\end{description}


\section{Nuclear Plant}

The nuclear plant is a nuclear power cell factory. It transforms a chunk of uranium ore into a nuclear power cell fully charged and ready for use.

\begin{description}
    \item[Requirement for building:] Any flat surface at least 10-meter wide
    \item[Input:] Uranium ore
    \item[Output:] Nuclear power cell
    \item[Power source:] None
    \item[Duration of operation:] 30 seconds
    \item[Object category:] \texttt{NuclearPlant}
\end{description}


\section{Converter}

This building was designed to convert chunks of titanium ore into usable cubes of titanium. All you need to do is place a chunk at the center of the platform and the converter takes care of the rest.

\begin{description}
    \item[Requirement for building:] Any flat surface at least 6-meter wide
    \item[Input:] Titanium ore
    \item[Output:] Titanium
    \item[Power source:] None
    \item[Duration of operation:] 15 seconds
    \item[Object category:] \texttt{Converter}
\end{description}


\section{Derrick}

The derrick is intended for the extraction of raw materials. In order to determine the best site for a derrick to be erected, a sniffer should be used to prospect the subsoil. The marks it lays down provide information about the specifics of what the derrick will have to extract:

a red cross $\rightarrow$ titanium ore

a yellow circle $\rightarrow$ uranium ore

\begin{description}
    \item[Requirement for building:] Any flat and geologically adequate surface at least 2.5-meter wide
    \item[Input:] None
    \item[Output:] Chunks of titanium or uranium ore
    \item[Power source:] None
    \item[Duration of operation:] 10 seconds for titanium, 30 seconds for uranium
    \item[Object category:] \texttt{Derrick}
\end{description}


\section{Power Captor}

The power captor acts as both a lightning conductor and a power converter. It offers a protection against the perilous lightning bolts of magnetic storms within a radius of 50 meters. Additionally, when the captor is hit by lightning, all bots and power cells placed underneath, with the exception of the nuclear power cells, find themselves re-energized. Note that your spaceship can also act as a lightning conductor. On the other hand it will obviously not recharge power cells. To visualize the zone shielded by a power captor, select it then hit the <range> button. Red dots outline the circular zone for 20 seconds. In the vastness of space, the planets subjected to magnetic storms are seldom. Among the ones you'll be visiting, only Orpheon appears to apply.

\begin{description}
    \item[Requirement for building:] Any flat surface at least 10-meter wide
    \item[Input:] Empty power cell
    \item[Output:] Recharged power cell
    \item[Power source:] Lightning bolts
    \item[Duration of operation:] 1 second
    \item[Object category:] \texttt{PowerCaptor}
\end{description}


\section{Vault}

This dome-shaped building was conceived by the first expedition. Its purpose was to offer protection to a new type of advanced bot. In order for it to be safe from insect attacks, the vault was locked by 4 keys. Only after you find the 4 keys and place each one of them on their corresponding slots will you be given access to the bot.

\begin{description}
    \item[Requirement for building:] Cannot be built
    \item[Input:] 4 keys
    \item[Output:] A bot that was hidden inside
    \item[Duration of operation:] \todo{Check}?
    \item[Object category:] \texttt{Vault}
\end{description}
