
\chapter{Functions}
\label{chap:cbot_functions}

\section{Function Declaration}


\section{Formal Parameters}


\section{Return Type}
Each function has a return type. Return type is a type of value that is returned by function. For example, function defined to return \texttt{int} will return values of type \texttt{int}. There is also \texttt{void} type which means that function does not return any value.

Value is returned from function using \texttt{return} statement. \texttt{return} also terminates execution of code inside this function. If you want to return a value, simply write:

\begin{lstlisting}
return value;
\end{lstlisting}

In case of \texttt{void} functions, you simply write:

\begin{lstlisting}
return;
\end{lstlisting}


\section{Function Body}
Function body is simply a list of statements within a block. Variables defined within function are local to the function and are not accessible outside.


\section{Overloading}
Functions can be overloaded, that is you can define multiple functions with the same name as long as they have different format parameters. For example, you can define multiple $max$ functions and even call each other, like in code below.

\begin{lstlisting}
float max(float x, float y)
{
    if(x > y) return x;
    else return y;
}

float max(float x, float y, float z)
{
    return max(max(x, y), z);
}
\end{lstlisting}
\todo[inline,color=green]{This is really broken in Gold right now -krzys\_h}


\section{Main Functions}
Every executable program has to contain a main function that will be run. This main function has to be properly marked using special keyword: \texttt{extern}. Function marked this way will also serve as a label on program list.

An example of main function is shown in the listing below.

\begin{lstlisting}
extern void object::Program()
{
    // code goes here
}
\end{lstlisting}

A function marked as \texttt{extern} cannot be marked as \texttt{public}.


\section{Public Functions}
A public function is a special function that can be accessed by other robots. To make function public you need to add keyword \texttt{public} before it's declared type.

An example of public function is shown below.

\begin{lstlisting}
public float sum(float[] numbers)
{
    float result = 0;
    int n = sizeof(numbers);
    
    for(int i=0; i<n; i++)
    {
        result += numbers[i];
    }
    
    return result;
}
\end{lstlisting}
