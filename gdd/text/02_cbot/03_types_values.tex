
\chapter{Types, Values, and Variables}
\label{chap:cbot_types}


\section{Built-in Types}

\subsection{Void type (\texttt{void})}
This is special type that has no actual valid values. It is only used to mark functions as not returning a value.


\subsection{Logical Type (\texttt{bool} and \texttt{boolean})}
This is a logical type which is used in conditions and comparison expressions. There are only two valid values for this type: \texttt{false} and \texttt{true}.

\texttt{bool} and \texttt{boolean} are two equivalent keywords for the same type. \texttt{bool} comes from languages such as C++ and C\#, while \texttt{boolean} is a keyword for logical type in Java language. You can use whichever keyword you want, however it is preferred to use one keyword in code. \texttt{boolean} was not documented in CBot manual.
\todo[inline]{GDD should describe what we WANT the game to be, not how it WAS made by Epsitec -krzys\_h}
\todo[inline]{We will likely change many things when we rewrite CBot interpreter. But before that happens, I think we should document what we have. -tomaszkax86}


\subsection{Integer Type (\texttt{int})}
This type represents integer numbers. Internally, \texttt{int} is a 32-bit signed integer. For this reason variables of this type can hold values in range $[-2^{31}, 2^{31}-1]$ or $[-2 147 483 648, 2 147 483 647]$.


\subsection{Floating Point Type (\texttt{float})}
Floating point type is used to represent real numbers. Internally it is a single-precision floating point number as specified in IEEE 754.


\subsection{String Type (\texttt{string})}
String type represents a sequence of characters.
\todo[inline]{Accessing characters like an array}


\subsection{Point Type (\texttt{point})}
\texttt{point} is a built-in class type for storing point coordinates. Each point has three \texttt{float} values named
\texttt{x}, \texttt{y}, and \texttt{z}.

Class \texttt{point} has definition that is equivalent to code below:

\begin{lstlisting}[style=cbot]
public class point
{
    public float x, y, z;
    
    void point()
    {
        this.x = 0;
        this.y = 0;
        this.z = 0;
    }

    void point(float x, float y)
    {
        this.x = x;
        this.y = y;
        this.z = 0;
    }
    
    void point(float x, float y, float z)
    {
        this.x = x;
        this.y = y;
        this.z = z;
    }
}
\end{lstlisting}


\subsection{Object Type (\texttt{object})}
Class \texttt{object} is a built-in class that represents in-game objects. Instances of this class can not be created in code, only instantiated using various instructions, such as \texttt{radar()} and \texttt{retobject()}.
\todo[inline]{Documentation of members}



\section{Variables}
A variable is a storage location in memory that has specified type and identifier. Type can be any of the built-in types or user-created class. Variable identifier is a valid name as specified in section \ref{sec:identifiers}.

Variable has an associated value of given type that can be changed using assignment operator, prefix or postfix decrementation (\texttt{-{}-}) and incrementation (\texttt{++}) statements.

Declaration of variable requires specifying a type and name of variable, and optionally initialization value. Examples of valid declarations are shown below:

\begin{lstlisting}
int number;
float x, y;                // you can declare multiple variables of the same type in one declaration
string text = "Some text";     // you can initialize variable with some value
\end{lstlisting}


\section{Arrays}
An array is a special type of variable that is a continuous region in memory for storing multiple values. Each array, like all variables, has associated type and identifier, but also length. Length of an array is a number of values it can store.

Arrays are declared similarly to ordinary variables, except you add \texttt{[]} in declaration. There are few ways to declare arrays. Valid declarations of array of type \texttt{int} are shown below:

\begin{lstlisting}
int[] identifier;
int identifier[];
int identifier[3];      // declare an array with length 3
\end{lstlisting}

Array elements are integer numbers from $0$ to $length-1$, where $length$ is a length of an array. Individual elements of given array can be accessed using indexer syntax with \texttt{[ ]}. Syntax is as follows:

\begin{lstlisting}
identifier[expression]
\end{lstlisting}

Where $identifier$ is a name of array and $expression$ specifies index to element within an array.

Arrays in CBot have a special property: they grow automatically when necessary. This means that, unlike in most programming languages, you can create an array of \texttt{int}s with length 3 and save to element $3$ (counting from 0) which will cause the array to grow to length 4.


\section{Conversion}
